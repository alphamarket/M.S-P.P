\documentclass[10pt,a4paper]{article}
% for margining standards
\usepackage[left=3cm,right=3cm,top=3cm,bottom=3cm]{geometry}
% for counting references as a section
\usepackage[numbib,notlof,notlot,nottoc]{tocbibind}
% useful packages
\usepackage{
                graphicx, setspace, fontspec, caption,
                subcaption, float, polyglossia, rotating,
                lscape, pdflscape, indentfirst, tocloft,
                multirow, currfile, mathtools, enumitem
            }
% paragraph related package
\usepackage[parfill]{parskip}
% use bzar font(THIS MUST BE LOADED BEFORE XePerian PACKAGE)
\setmainfont{BZar.ttf}
% the dear XePersian package
\usepackage{xepersian}
%
% General settings goes here.
%
% lines space
\renewcommand{\baselinestretch}{1.6}
% paragraph first line indention
\setlength{\parindent}{1cm}
% paragraph spacing
\setlength{\parskip}{1em}
% set graphics' path
\graphicspath{ {images/} }
% make table of content dotted
\renewcommand{\cftsecleader}{\cftdotfill{\cftdotsep}}
% define a new command as {half-space} in english
\newcommand{\halfspace}{\hspace{0pt}}
% define a new command as {half-space} in persian
\newcommand{\نیمفاصله}{\halfspace}
% define a shortcut for half-space in general
\renewcommand{\ }{\halfspace}
% define a new command for ease of use for rendering reference
\newcommand{\renderref}[1] { \begingroup \let\clearpage\relax \include{#1} \endgroup }
% set no space between lists' items
\setlist{nolistsep}
%
% DOCUMENT BEGIN
%
\begin{document}
\title{
    \includegraphics[width=0.25\textwidth]{iut}\\\vspace{30pt}
    کاربرد سنجش فازی\ بودن\\در بررسی دسته\ بندی ریسک در بیمه    
}
\author{داریوش حسن\ پور آده}
\date{۹۳۰۸۱۶۴}
\maketitle
\null
\vfill
\thispagestyle{empty}
\setcounter{page}{0}
\newpage
این مقاله\
\cite{THEPAPER}
نشان می\ دهد که چگونه می\ توان از سنجش فازی\ بودن\زیرنویس{\lr{Measure Of Fuzziness}} برای دسته\ بندی\زیرنویس{\lr{Classification}} ریسک برای اهداف بیمه\ ای(مثلا بیمه\ ی عمر) استفاده کرد. ریسک داده شده به عنوان یک ریسک فازی ترجیحی توصیف می\ شود، بعد میزان فازی\ بودن این ریسک برای دسته\ بندی آن ریسک، مورد اندازه\ گیری واقع می\ شود.
در این مقاله چندین نوع سنجش میزان فازی\ بودن را مورد مقایسه قرار می\ دهد. که شامل میزان آنتروپی، میزان فاصله، میزان ضرب\ بدیهی\زیرنویس{\lr{Axiomatic Product}} می\ شود؛ همچنین کابردهای آنها نیز در دسته\ بندی ریسک بیان می\ شود.\بند
تئوری مجموعه\ های فازی بخوبی توانسته در مسائل تصمیم\ گیری\ های تقریبی کاربردهایی بیابد. طبقه\ بندی ریسک را که به عنوان طبقه\ بندی خطراتی که یک بیمه\ نامه می\ تواند برای سازمان بیمه\ کننده داشته باشد را تعریف کنیم؛ که با توجه به احتمال درخواست مطالبه توسط مشتری و با توجه به اندازه آن مطالبه\ ها یکی از مفاهیم بنیادی دانش\ آماری می\ باشد.\\
در روش\ های معمولی دسته\ بندی ریسک به عنوان مثال می\ گویند که میزان کلسترول خون مشتری در صورتی که از ۲۰۰ واحد بیشتر نباشد می\ تواند مورد خوبی برای جذب از سمت شرکت بیمه باشد، در حالی که میزان ۲۰۱ واحد طبق این طبقه\ بند مجاز نیست! این دقت ممکن است شرکت بیمه را نهایتا به موقعیت ٬ضد انتخابی\زیرنویس{\lr{Anti-Selection}}٬ سوق دهد. که دو نمونه موجود در لبه\ ی خط جداکننده به دسته\ های متفاوتی دسته\ بندی می\ شوند درحالی که آنها در ویژگی\ های پایه\ ای ریسک مشابه هم هستند. روشی ارائه شده که پیشنهاد می\ دهد که بجای اینکه از راه\ حل قبلی(مبنی بر جداسازی قطعی نمونه\ ها) بیاییم ریسک\ ها را با استفاده از کلاسترهای کریسپ\زیرنویس{\lr{Crisp}} یا فازی دسته\ بندی کنیم و سپس بیاییم نمونه\ ها(افراد/بیمه\ نامه) را دسته\ بندی کنیم.\بند
در بسیاری از مواقع پیشاپیش می\ دانیم که چه مشخصاتی یک ریسک می\ تواند داشته باشد. هر متقاضی می تواند متناسب با خطر ترجیحی ٬ایده آل٬ نسبت به اندازه\ گیری ویژگی\ های مشخصات مقایسه شود سپس یک درجه\ ی عضویت می\ تواند به هریک از اندازه\ گیری\ ها تخصیص داده شود؛ این روند یک برداری از مقادیر اندازه\ گیری شده فازی را تولید می\ کند، با اندازه\ گیری میزان فازی\ بودن آن اندازه\ گیری\ ها می\ توان طبقه\ بندی جدیدی را تعیین کرد.\بند
اندازه\ گیری فازی\ بودن درجه\ ی فازی بودن آن مجموعه فازی را نشان میدهد. علاوه بر اندازه\ گیری آنتروپی و فاصله\ ی مجموعه\ ی فازی نسبت به مکمل آن؛ روش دیگری بنام اندازه\ گیری ضرب\ بدیهی که ویژگی\ ای را بیان می\ کند که یک اندازه\ گیر فازی\ بودن باید دارا باشد. و نشان می\ دهد که برای جهان متناهی
$U = \{x_1, x_2, \ldots, x_n\}$
و زیرمجموعه\ ی فازی
$\tilde{E}$
با تابع عضویت
$\mu : U \rightarrow \left[0, 1\right]$
که اندازه\ گیری منحصر به فرد توسط ضرب\ بدیهی توسط رابطه\ ی
\ref{eq:M}
بدست می\ آید.
\begin{equation}
M\left(\tilde{E}\right) = \sum_{i=1}^{n} \mu\left(x_i\right)\left(1-\mu\left(x_i\right)\right)
\label{eq:M}
\end{equation}
که به رابطه\ ی
\ref{eq:M}
اندازه\ گیری ضرب\ بدیهی می\ گویند.\بند
این مقاله برای اینکه کاربرد سنجش فازی\ بودن را در دسته\ بندی بیان کند مثال زیر را مطرح کرده است:\\
از میان مشخصات مورد نظر برای ریسک ترجیحی(مطلوب) جهت سرمایه\ گذاری/جذب مشتری دارای این ویژگی\ ها توسط شرکت بیمه، می\ توانیم به موارد زیر اشاره کنیم:

\begin{enumerate}
\item حداکثر میزان کلسترول خون برابر ۲۰۰ \lr{mg/dl}
\item حداکثر فشار خون برابر ۱۳۰
\item نسبت وزن بدن به وزن استاندارد پیشنهاد شده(که تابعی از قد می\ باشد) ۱ باشد
\item سیگار نکشد
\end{enumerate}
به ازای هر نمونه(مشتری) می\ توانیم درجه\ ای(مابین ۰ و ۱) برای آن نمونه تخصیص دهیم که نشان دهنده\ ی این است که چقدر موارد ۱\نقاط ۴ در آن مورد صادق است. بنابراین هریک از مشتری\ ها را با برداری با میزان\ های
$<s_1, s_2, s_3, s_4>$
نمایش می\ دهیم سپس درجه\ ی کلی عضویت در آن ریسک مورد نظر(مطلوب) بدست می\ آید.(با استفاده از استنتاج فازی یا اپراتور تقاطع فازی\زیرنویس{\lr{Fuzzy Intersection Operator}})\بند
این مساله باعث تبدیل مجموعه\ ای از مشتری\ های بیمه به عناصر جهان برای یک ریسک ترجیحی فازی می\ شود که هریک از این افراد با یک درجه عضویت نسبت به گروه ریسک ترجیحی ایده\ ال مقدار دهی می\ شوند. بدیهی است که هدف اصلی دسته\ بندی این است که گروه پرریسک را از گروه کم ریسک جدا کنند؛ بنابرین ما می\ خواهیم که زیرمجموعه\ ای از ٬ریسک\ های مناسب\زیرنویس{\lr{Good Risks}}٬ را از جهان بابیم که میزان فازی\ بودن آن کمینه باشد.\بند
برای این هدف(پیدا کردن زیرمجموعه\ ای از جهان که میزان فازی\ بودن آنها کمینه باشد) معیار آنتروپی برای سنجش فازی\ بودن زیرمجموعه\ ها در این مساله(بیمه) مناسب نیست زیرا که همیشه مقداری کمینه برای مجموعه\ ی ریسک ترجیحی یک-عضو می\ دهد. معیار سنجش فاصله نتایج خوبی با دسته\ بندی آن افرادی که نزدیک به ریسک ایده\ آل ترجیحی در یک دسته رو بقیه در دسته\ های دیگر ارائه می\ دهد.\بند
این مقاله سعی داشته نشان دهد که سنجش فازی\ بودن می\ تواند به عنوان ابزاری جهت دسته\ بندی ریسک برای یک تعهدنامه بیمه\ ای مورد استفاده قرار دهد. و همچنین سنجش فاصله\ ای می\ تواند برای تولید یک مجموعه ریسک ترجیحی فازی که مقدار فازی\ بودن آن در جهان کمینه است مورد استفاده قرار گیرد و در حالی که آنتروپی و ضرب\ بدیهی ابزارهایی برای ارزیابی یک مجموعه فازی از ریسک\ های ترجیحی هستند.
\vspace{5em}
\\{\LARGE مرجع}
\renderref{reference}
\end{document}
