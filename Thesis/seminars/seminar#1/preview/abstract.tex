\documentclass[10pt,a4paper]{article}
% for margining standards
\usepackage[left=3cm,right=3cm,top=3cm,bottom=3cm]{geometry}
% for counting references as a section
\usepackage[numbib,notlof,notlot,nottoc]{tocbibind}
% useful packages
\usepackage{
                graphicx, setspace, fontspec, caption,
                subcaption, float, polyglossia, rotating,
                lscape, pdflscape, indentfirst, tocloft,
                multirow, mathtools, currfile
            }
% paragraph related package
\usepackage[parfill]{parskip}
% use bzar font(THIS MUST BE LOADED BEFORE XePerian PACKAGE)
\setmainfont{BZar.ttf}
% the dear XePersian package
\usepackage{xepersian}
%
% General settings goes here.
%
% lines space
\renewcommand{\baselinestretch}{1.5}
% paragraph first line indention
\setlength{\parindent}{1cm}
% paragraph spacing
\setlength{\parskip}{1em}
% set graphics' path
\graphicspath{ {images/} }
% make table of content dotted
\renewcommand{\cftsecleader}{\cftdotfill{\cftdotsep}}
% define a new command as {half-space} in english
\newcommand{\halfspace}{\hspace{0pt}}
% define a new command as {half-space} in persian
\newcommand{\نیمفاصله}{\halfspace}
% define a shortcut for half-space in general
\renewcommand{\ }{\halfspace}
% define a new command for ease of use for rendering reference
\newcommand{\renderref}[1] { \begingroup \let\clearpage\relax \include{#1} \endgroup }
%
% DOCUMENT BEGIN
%
\begin{document}
\title{سینماتیک چهارپره و یادگیری کنترل آن توسط\\یادگیری تقویتی}
\author{داریوش حسن\ پور}
\date{پاییز ۱۳۹۴}
\maketitle
چهارپره\زیرنویس{\lr{Quadrotor}} هلیکوپتر چندپره است که توسط چهار پره به پرواز و حرکت در می\ آید که به عنوان وسایل پره\ ای طبقه بندی می\ شوند.
در این سمینار به ارائه\ ی یک مدل سینماتیکی از سینماتیک چهارپره خواهیم پرداخت، سینماتیک شاخه‌ای از دانش مکانیک کلاسیک است که حرکت اجسام و سامانه‌ها(گروهی از اجسام) را بدون درنظرگرفتن نیروهای عامل حرکت بررسی می‌کند. در این سمینار به تحلیل نیروهای وارد بر چهارچوب چهارپره نیز پرداخته و در نهایت توضیحی در مورد معادلات حرکتی چهارپره که رفتار چهارپره را با توجه به پارامترهای آن شرح می\ دهد، داده می\ شود.
\end{document}
