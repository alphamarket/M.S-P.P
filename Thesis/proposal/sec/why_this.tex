%!TeX TS-program = XeLaTeX
%!TeX root = thesis.proposal.tex
\قسمت{ضرورت انجام، دیدگاه/روش و اهداف تحقیق پیشنهادی}
از آنجایی که به کاربردهای چهارپره\ ها روزبه\ روز افزوده می\ شود، امنیت پرواز این نوع از پهپادها به مساله\ ی جالب و مهمی تبدیل شده است، از جمله\ ی موارد امنیت پرواز می\ توان به کاهش احتمال برخورد با اشیا و سقوط اشاره کرد. همچنین علاوه\ بر استفاده\ ی \کادربی{روزمره\ ی} چهارپره بین افراد شهروند، چهارپره توجه زیادی را نیز از سمت جامعه دانشگاهی دریافت میکند. بدیهی است که طراحی و ساخت چهارپره نیازمند زمان و هزینه می\ باشد، حال اگر سامانه\ ی اجتناب از برخورد با موانع بروی چهارپره نصب باشد در حین کارباچهارپره(چه توسط فرد غیرخبره\ ی شهروند و چه در تحقیقات دانشگاهی) احتمال برخورد با اشیا و سقوط و خرابی چهارپره کاهش می\ یابد\زیرنویس{به عبارت دیگر افرادی که بعد از بنده قرار است بروی چهارپره\ ای که بنده بروی آن کار کرده\ ام تحقیق کنند،‌ خیالشان از بابت تصادف چهارپره با اشیا محیط اطرافشان(که می\ تواند موجب خسارات جانی یا مالی گردد) راحت خواهد بود.} و در نهایت کلام سامانه\ ای که قرار است در این تحقیق به پیاده\ سازی آن بپردازیم گامی در جهت افزایش امنیت پرواز چهارپره\ ها خواهد بود.\بند
علاوه\ بر ضرورت ایجاد امنیت پرواز در چهارپره\ ها، کارهایی قبلی موفقی\مرجع{borenstein1991vector, ulrich1998vfhplus, kim1992real, chakravarthy1998obstacle, ulrich2000vfh, roberts2007quadrotor, bouabdallah2007toward} در زمینه اجتناب از مانع که با بکارگیری چندین حسگر همپوشان تلاش به ایجاد نقشه\ ی موانع محیط می\ کردند، نتایج خوبی گرفته\ اند. از سوی دیگر انتگرال\ های فازی\مرجع{grabisch1995fuzzy} که عمکرد خوبشان در ترکیب اطلاعات چند عامل\ اطلاعاتی و به دست دادن اطلاعات با کیفیت جهت تصمیم\ گیری بهتر(با در نظر گرفتن همه\ ی جوانب داده\ ها) در گذشته ثابت شده است\مرجع{sousa2002fuzzy, grabisch1995fuzzy, tehrani2012preference, liou2014fuzzy, labreuche2003choquet, feng2010hybrid, geldermann2000fuzzy, grabisch2000application}؛ باتوجه به ماهیت ذاتی \کادربی{انتگرال\ های} فازی در ترکیب اطلاعات، به\ نظر می\ رسد که با ترکیب اطلاعات حسگرهای همپوشان نصب شده بروی چهارپره توسط انتگرال\ های فازی می\ توان نتایج بهتری در اجتناب از موانع گرفت ولی تاکنون تلاشی جهت ترکیب اطلاعات حسگرها با استفاده از انتگرال\ های فازی صورت نگرفته است، که همین دلیل به جذابیت مساله می\ افزاید. در این تحقیق \زیرخط{درصورتی که به تعداد کافی حسگر داشته باشیم}\زیرنویس{که به میزان بودجه،‌ سخت\ افزارهای دراختیار و کیفیت آن\ ها بستگی دارد.} که حداقل بین اطلاعات برخی از حسگرها نسبت به موقعیت موانع همپوشانی داشته باشیم، علاوه بر \کادربی{پیاده\ سازی} روشی برای اجتناب از مانع، به بررسی این مساله که «آیا ترکیب اطلاعات حسگرها برای اجتناب از موانع با استفاده از انتگرال\ های فازی می\ تواند موثر باشد یا خیر؟» نیز خواهیم پرداخت.